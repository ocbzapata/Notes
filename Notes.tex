%Prefacio
\include{Prefacio}

\title{Category Theory and \\ Computational Complexity}
\author{Marco Larrea \and Octavio Zapata}

\begin{document}
\maketitle
A first-order dependence logic $D$ is a class which consists of all $D$-definable properties where 
$D := FO + \mu.\bar{t}$ and $\mu.\bar{t}$ denotes that term $t_{|\bar{t}|}$ is functionally dependent on 
$t_{i}$ for all $i\leq |\bar{t}|$. The model class $FO$ is as always defined as the class of models of all 
first-order sentences and $\mu.\bar{t}$ is interpreted as a recursively generated tuple of terms which we 
naturaly identify with the set $[|\bar{t}|] := \{1,2,\dots,|\bar{t}|\}$. $D$ sentences are capable to characterise 
variable dependence and in general they are proven to be as expressive as sentences of the second order $\Sigma_1^1$ 
fragment. The intuitionistic version $ID$ has the same expressive power as full $SO$. It is a fact that $MID$-model 
checking is $PSPACE$-complete where $MID$ is the intuitionistic implication fragment of the modal dependence logic 
$MD$ which contains at least two modifiers. 

%Referencias
\nocite{*}
\bibliographystyle{alpha}
\bibliography{paper}

\end{document}
